\anonsection{Введение}

3D графика нашла своё применение в множестве областей нашей жизни. Она широко применяется в рекламе, в науке, в различных видах творчества, но особое место графика заняла в кинематографе, мультипликации и игровой индустрии.

Зачастую бывает недостаточно просто создать модель, наложить на нее текстуры, добавить света и отрендерить, а встаёт вопрос придания ей динамики. Созданную модель, ровно как и объекты в сцене, можно перемещать в пространстве, вращать, деформировать, налету менять текстуру или даже физические свойства. Отдельной от всех этих процессов частью стоит анимирование персонажей.

Анимирование персонажа, как человекоподобного, так и животного, достаточно трудоёмкий процесс, в который входят:
\begin{itemize}
	\item создание модели;
	\item оснащение модели костями и их настройка;
	\item раскраска весов (weight painting);
	\item непосредственно сам процесс анимирования.
\end{itemize}

Обычно, процесс непосредственного анимирования делится на крупное, например движение рук, ног, туловища и головы, и более мелкое и кропотливое, напимер анимация пальцев на руках или лицевая анимация. И, если для достижения хорошего результата в "большом" анимировании не требуется серьезных затрат (если, конечно, не требуются сложные движения, например танцевальные), то вот мелкая моторика и лицевая анимация требуют уже больших усилий для нужного результата.




\clearpage
