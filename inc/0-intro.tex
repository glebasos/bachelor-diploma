\anonsection{Введение}

3D графика нашла своё применение в множестве областей нашей жизни. Она широко применяется в рекламе, в науке, в различных видах творчества, но особое место графика заняла в кинематографе, мультипликации и игровой индустрии.
 
Зачастую бывает недостаточно просто создать модель, наложить на нее текстуры, добавить света и отрендерить, а встаёт вопрос придания ей динамики. Созданную модель, ровно как и объекты в сцене, можно перемещать в пространстве, вращать, деформировать, налету менять текстуру или даже физические свойства. Отдельной от всех этих процессов частью стоит анимирование персонажей.

Анимирование персонажа, как человекоподобного, так и животного, достаточно трудоёмкий процесс, в который входят:
\begin{itemize}
	\item создание модели;
	\item оснащение модели костями и их настройка;
	\item раскраска весов (weight painting);
	\item непосредственно сам процесс анимирования.
\end{itemize}

Обычно, процесс непосредственного анимирования можно условно поделить на крупное, например, движение рук, ног, туловища и головы, и более мелкое и кропотливое, например анимация пальцев на руках или лицевая анимация. И, если для получения хорошего результата в "большом" анимировании не требуется серьезных затрат (если, конечно, не требуются сложные движения, например танцевальные), то мелкая моторика и лицевая анимация требуют уже больших усилий для достижения желаемого результата.

Чтобы лицо смотрелось живо и органично, необходимо чтобы каждая кость и каждая мышца, вели бы себя так, как и на реальном, живом образце. А так как зачастую таких мышц много, то процесс создания лицевой анимации становится одним из самых времязатратных процессов при анимировании.

Для того, чтобы уменьшить затраты на этот процесс, были созданы технологии захвата движения. Захват движений (по английски - motion capture) позволяет перенести движения реального человека на персонажа один к одному, что существенно упрощает процесс анимирования.

Две основные системы при захвате движений - маркерные и безмаркерные. В маркерных системах компьютер сводит полученные с камер данные о маркерах - отметках на человеке и передает их на трехмерную модель, когда как безмаркерные системы делают это без использования дополнительных ориентиров, с помощью технологий компьютерного зрения и распознавания образов.

Представленная в пакете Blender система захвата движений является маркерной, а также не позволяет создавать анимацию в режиме реального времени, поэтому задачей данной работы стало создание безмаркерной системы захвата движений реального времени, которая бы смогла упростить и ускорить процесс создания лицевой анимации.

\clearpage
